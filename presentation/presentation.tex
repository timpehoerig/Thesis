\documentclass[aspectratio=169]{beamer}
\usetheme{Boadilla}
\useoutertheme{infolines}
\setbeamertemplate{navigation symbols}{}

\usepackage[utf8]{inputenc}
\usepackage[T1]{fontenc}
\usepackage{graphicx}
\usepackage{mathtools}
\usepackage{amssymb}
\usepackage{amsmath}
\usepackage{amsfonts}
\usepackage{hyperref}
\usepackage{bbm}
\usepackage{newunicodechar}
\usepackage{minted}
\usepackage{xcolor}
\usepackage{tikz}


\newcommand{\inl}[1]{\mintinline[breaklines, breakafter=→, escapeinside=||]{java}{#1}}
\newcommand{\classheader}{\text{class C<} \overline{\text{X}} \vartriangleleft \overline{\text{N}} \text{>} \vartriangleleft \text{N}}
\newcommand{\xn}{\text{<} \overline{\text{X}} \vartriangleleft \overline{\text{N}} \text{>}}
\newcommand{\yp}{\text{<} \overline{\text{Y}} \vartriangleleft \overline{\text{P}} \text{>}}
\newcommand{\ot}[1]{\overline{\text{#1}}}
\newcommand{\vv}[1]{\text{<} #1 \text{>}}
\newcommand{\txn}[3]{\text{[#1/#2]#3}}
\newcommand{\tsxsn}[3]{\text{[}\ot{#1}/\ot{#2}\text{]#3}}
\newcommand{\tsxsns}[3]{\text{[}\ot{#1}/\ot{#2}\text{]}\ot{#3}}
\newcommand{\tsxsnd}[5]{\text{[}\ot{#1}/\ot{#2}\text{]}\text{[}\ot{#3}/\ot{#4}\text{]#5}}
\newcommand{\tsxsnsd}[5]{\text{[}\ot{#1}/\ot{#2}\text{]}\text{[}\ot{#3}/\ot{#4}\text{]}\ot{#5}}
\newcommand{\cm}{\text{C} \xn \text{.m}}
\newcommand{\oto}[1]{\overline{#1} \to #1}
\newcommand{\mtype}{\text{mtype} ( \text{m}, \ \text{N}, \ \prod )}
\newcommand{\fresh}[1]{#1 \ \text{fresh}}
\newcommand{\texpr}[1]{\text{TYPEExpr}((\prod; \overline{\eta}), #1)}
\newcommand{\nvt}[2]{\text{#1<}\ot{#2}\text{>}}
\newcommand{\class}[2]{\classheader \ \{ \ \ot{#1} \ \ot{#2} ; \ [\text{K}] \ \ot{M} \ \} }
\newcommand{\cand}{\text{C} \cup}
\newcommand{\cxnm}[4]{\text{#1} \text{<} \text{#2} \vartriangleleft \text{#3} \text{>}. \text{#4}}


\title[Type Inference for FGJ]{An Implementation of Type Inference for\\Featherweight Generic Java}
\institute[Uni Freiburg]{Chair of Programming Languages, University of Freiburg}
\author{Timpe Hörig}

\begin{document}

\begin{frame}
    \titlepage
\end{frame}

\section{Featherweight Generic Java}

\begin{frame}[fragile]
    \frametitle{Syntax}
    \begin{align*}
        \text{T} &:= \text{X} \ | \ \text{N}\\
        \\
        \text{N} &:= \text{C} \vv{\ot{T}}\\
        \\
        \text{L} &:= \text{class} \ \text{C} \xn \ \vartriangleleft \ \text{D} \{ \ \ot{T} \ \ot{f} \ [\text{K}] \ \ot{M} \ \}\\
        \\
        \color{red}\text{K} &\color{red}:= \text{C}(\ot{T} \ \ot{f})\{ \text{super}(\ot{f}); \ \text{this}. \ot{f} = \ot{f}; \} \\
        \\
        \text{M} &:= \color{red} \xn \text{T} \ \color{black} \text{m}(\color{red} \ot{T} \ \color{black} \ot{x}) \ \{\text{return e}\}\\
        \\
        \text{e} &:= \text{x} \ | \ \text{e.f} \ | \ \text{e.} \color{red} \text{<} \ot{T} \text{>} \color{black} \text{m(}\ot{e}\text{)} \ | \ \text{new C(}\ot{e}\text{)} \ | \ \text{(C)e}
    \end{align*}
\end{frame}

\begin{frame}[fragile]
    \frametitle{AST}
    \begin{minted}[escapeinside=||]{python}
        @dataclass
        class ClassDef:
            name: str,
            superclass: Type,
            typed_fields: FieldEnv,
            methods: list[MethodDef]
    \end{minted}
\end{frame}

\begin{frame}[fragile]
    \frametitle{Parser}
    \begin{minted}[escapeinside=||]{python}
        # grammar rules
    
        identifier: ...
    
        variable: identifier
    
        [...]
    
        # function for shaping
    
        def variable(tuple_of_elements_of_variable_rule):
            (name, ) = tuple_of_elements_of_variable_rule
            return Variable(name)
    \end{minted}
\end{frame}

\begin{frame}[fragile]
    \frametitle{extended Syntax}
    \begin{align*}
        \text{T} &:= \text{a} \ | \ \text{X} \ | \ \text{N} &\text{type variable, type parameter, non type variable} \\
        \text{N} &:= \text{C} \vv{\ot{T}} &\text{class type (with type variables)}\\
        \text{sc} &:= \text{T} < \text{U} \ | \ \text{T} = \text{U} &\text{simple constraint: subtype or equality}\\
        \text{oc} &:= \{ \{ \ot{sc}_1 \}, \ ..., \{ \ot{sc}_n \} \} &\text{or-constraint}\\
        \text{c} &:= \text{sc} \ | \ \text{oc} &\text{constraint}\\
        C &:= \{ \ot{c} \} &\text{constraint set}\\
        \lambda &:= \cm : \yp \oto{\text{T}} &\text{method type assumption}\\
        \eta &:= \text{x} : \text{T} &\text{parameter assumption}\\
        \sprod{0.7} &:= \sprod{0.7} \cup \overline{\lambda} &\text{method type environment}\\
        \Theta &:= ( \sprod{0.7}; \overline{\eta} )
    \end{align*}
\end{frame}

\begin{frame}[fragile]
    \frametitle{Type Inference}
    \begin{align*}
        \text{FJType}&\text{Inference}(\sprod{0.7}, \ \classheader \{ ... \}) = \\
        \text{let} & \ (\overline{\lambda}, \ C) \quad \quad \quad = \text{FJType}(\sprod{0.7}, \classheader \{ ... \}) \\
        & \ (\sigma, \yp) = \text{Unify}(C, \{ \overline{\text{X}} <: \overline{\text{N}} \} \cup \{ \zq \ | \\
        & \quad \quad \quad \quad \quad \quad \quad \quad \quad \quad \quad \quad \quad (\cm : \zq \oto{T} ) \in \overline{\lambda} \} ) \\
        \text{in} \ & \sprod{0.7} \cup \ \{( \cm \ : \ \yp \ \oto{\sigma(a)}) \ | \\
        & \quad \quad \quad \quad \quad \quad \quad \quad \quad \quad \quad \quad \quad (\cm : \oto{a} ) \in \overline{\lambda} \}
    \end{align*}
\end{frame}

\begin{frame}[fragile]
    \frametitle{FJType}
    \begin{align*}
        \text{FJType}(\sprod{0.7}, & \ \classheader \{ \ot{T} \ \ot{f}; \ \ot{M} \}) = \\
        \text{let} \ & \overline{a}_m \ \text{be fresh type variables for each m} \in \ot{M} \\
        & \overline{\lambda}_0 = \{\cm : \ \yp \ot{T} \to a_m \\
        & \quad \quad \quad \quad \quad \quad | \ \text{m} \in \ot{M}, \ \mtype = \yp \ot{T} \to \text{T} \}\\
        & C_0 = \{ a_m < \text{T} \ | \ \text{m} \in \ot{M}, \ \mtype = \yp \oto{T} \}\\
        & \overline{\lambda}' = \{ ( \cm : \ \overline{a} \to a_m ) \\
        & \quad \quad \quad \quad \quad \quad | \ \text{m} \in \ot{M}, \ \mtype \ \text{not defined}, \ \fresh{\overline{a}} \}\\
        & C_m = \{\{ a_m < \text{Object}, \overline{a} < \ot{Object} \} \ | \ ( \cm : \ \overline{a} \ \to a_m) \in \overline{\lambda}' \} \\
        & \sprod{0.7} = \sprod{0.7} \cup \overline{\lambda}' \cup \overline{\lambda}_0\\
    \text{in} \ &(\sprod{0.7}, C_0 \cup C_m \cup \bigcup_{ \text{m} \in \ot{M} } \text{TYPEMethod}(\sprod{0.7}, C \vv{\ot{X}}, \text{m}))\\
    \end{align*}
\end{frame}

\begin{frame}[fragile]
    \frametitle{TYPEMethod}
    \begin{align*}
        \text{TYPEMethod}&(\sprod{0.7}, \ \text{C<} \ot{X} \text{>}, \ \text{m} ( \ot{x} ) \{ \ \text{return e}; \} ) = \\
        \text{let} \ &\yp \oto{\text{T}} = \sprod{0.7} ( \cm ) \\
        & ( \text{R}, C ) = \text{TYPEExpr} (( \sprod{0.7}; \{ \text{this} : \text{C<} \ot{X} \text{>} \} \cup \{ \ot{x} : \ot{T} \} ) , \text{e} ) \\
        \text{in} \ \  & C \cup \{ \text{R} < \text{T} \}\\
    \end{align*}
\end{frame}

\begin{frame}[fragile]
    \frametitle{TYPEExpr | Variable}
    \begin{align*}
        \texpr{\text{x}} = (\overline{\eta} (\text{x}), \emptyset)\\
    \end{align*}
\end{frame}

\begin{frame}[fragile]
    \frametitle{TYPEExpr | Field Lookup}
    \begin{align*}
        \texpr{e.f} &= \\
        \text{let} \ &( \text{R}, C_R ) = \texpr{e} \\
        &\fresh{a} \\
        &\text{c} = \text{oc} \{ \{ \text{R} < \text{C<} \ot{a} \text{>}, \text{a} = \tsxsn{a}{X}{T}, \ot{a} < \tsxsns{a}{X}{N} \ | \ \fresh{\ot{a}} \} \\
        & \quad \quad \quad \quad | \ \text{T f} \in \class{X}{N} \}\\
        \text{in} \ \ & (\text{a}, (C_R \cup \{ \text{c} \} ))
    \end{align*}
\end{frame}

\begin{frame}[fragile]
    \frametitle{Solved Form}
    \begin{align*}
        1.& \quad \text{a} < \text{b}\\
        \\
        2.& \quad \text{a} = \text{b}\\
        \\
        3.& \quad \text{a} < \text{C<} \ot{T} \text{>}\\
        \\
        4.& \quad \text{a} = \text{C<} \ot{T} \text{>} \ \text{with a} \notin \ot{T}\\
        \\
        \text{No variable is allowed to occur } & \text{twice on the left side of rules 3 and 4.}
    \end{align*}
\end{frame}

\begin{frame}[fragile]
    \frametitle{Step 1}
    \begin{align*}
        &\frac{\cand \{ \text{a} < \nvt{C}{T}, \ \text{a} < \nvt{D}{V} \} }{ \cand \{ a < \nvt{C}{T}, \ \nvt{C}{T} < \nvt{D}{V} \} } \quad \Delta \vdash \nvt{C}{X} <: \nvt{D}{N} &\text{match}\\
        \\
        &\frac{\cand \{ \nvt{C}{T} < \text{a}, \ \nvt{D}{V} < \text{a} \} }{ \cand \nvt{C}{T} < \nvt{D}{V}, \ \nvt{D}{V} < \text{a} } \quad \Delta \vdash \nvt{C}{X} <: \nvt{D}{N} &\text{match reverse}\\
        \\
        &\frac{\cand \{ \text{a} < \nvt{C}{T}, \ \text{b} <^* \text{a}, \text{b} < \nvt{D}{U}  \} }{ \cand \{ \text{a} < \nvt{C}{T}, \ \text{b} <^* \text{a}, \ \text{b} < \nvt{D}{U}, \ \text{b} < \nvt{C}{T} \} } &\text{adopt}\\
        \\
        &\frac{\cand \{ \nvt{C}{T} < \text{a}, \ \text{a} <^* \text{b}, \nvt{D}{U} < \text{b} \} }{ \cand \{\nvt{C}{T} < \text{a}, \ \text{a} <^* \text{b}, \ \nvt{D}{U} < \text{b}, \ \nvt{C}{T} < \text{b} \} } &\text{reverse adopt}
    \end{align*}
\end{frame}

\begin{frame}[fragile]
    \frametitle{Step 2}
    \begin{align*}
        1.& \quad \nvt{C}{T} < \nvt{D}{U} \quad \text{where C cannot be a subtype of D.}\\
        2.& \quad a < \nvt{C}{T}, \ a < \nvt{D}{U} \quad \text{where C cannot be a subtype of D and vice versa.}\\
        3.& \quad \nvt{C}{T} < a
    \end{align*}
\end{frame}

\begin{frame}[fragile]
    \frametitle{expandLB}
    \begin{align*}
        \text{expandLB} ( \nvt{C}{T} < a, \ a < \nvt{D}{U} ) = \{ \{ &a = \tsxsn{T}{X}{N} \} \\
         &| \ \Delta \vdash \nvt{C}{X} < \text{N}, \ \Delta \vdash \text{N} < \nvt{D}{P} \}\\
        \text{where} \ \ot{P} \ \text{is determined by } & \Delta \vdash \nvt{C}{X} < \nvt{D}{P} \ \text{and} \ \tsxsns{T}{X}{P} = \ot{U}\\
    \end{align*}
\end{frame}

\begin{frame}[fragile]
    \frametitle{Step 3}
    \begin{align*}
        \frac{C \cup \{ \text{a} = \text{T} \}}{ \txn{T}{a}{C} \cup \{ \text{a} = \text{T} \} } \quad \text{where a occurs in C but not in T}
    \end{align*}
\end{frame}

\begin{frame}[fragile]
    \frametitle{Step 4}
    \begin{align*}
        \text{If C'' has changed, then start again from Step 1.} 
    \end{align*}
\end{frame}

\begin{frame}[fragile]
    \frametitle{Step 5}
    \begin{align*}
        \frac{\cand \{ \text{C} \cup \text{a} < \text{b} \}}{ \txn{a}{b}{C} \cup \{ \text{b} = \text{a} \} } \quad &\text{sub elim}
        \\
        \\
        \frac{\cand \{ \text{a} = \text{a} \} }{ C } \quad &\text{erase}
    \end{align*}
\end{frame}

\begin{frame}[fragile]
    \frametitle{Step 6}
    \begin{align*}
        \sigma &= \{ \text{b} \to \tsxsn{Y}{a}{T} \ | \ (\text{b} = \text{T}) \in C_= \} \cup \{ \ot{a} \to \ot{Y} \} \cup \{ \text{b} \to \text{X} \ | \ (\text{b} < \text{X}) \in C_< \}\\
        \\
        \gamma &= \{ \text{Y} \vartriangleleft \tsxsn{Y}{a}{N} \ | \ (\text{a} < \text{N}) \in C_< \}\\
    \end{align*}
\end{frame}

\begin{frame}[fragile]
    \frametitle{Type Infernece}
    \begin{align*}
        \text{FJType}&\text{Inference}(\sprod{0.7}, \ \classheader \{ ... \}) = \\
        \text{let} & \ (\overline{\lambda}, \ C) \quad \quad \quad = \text{FJType}(\sprod{0.7}, \classheader \{ ... \}) \\
        & \ (\sigma, \yp) = \text{Unify}(C, \{ \overline{\text{X}} <: \overline{\text{N}} \} \cup \{ \zq \ | \\
        & \quad \quad \quad \quad \quad \quad \quad \quad \quad \quad \quad \quad \quad (\cm : \zq \oto{T} ) \in \overline{\lambda} \} ) \\
        \text{in} \ & \sprod{0.7} \cup \ \{( \cm \ : \ \yp \ \oto{\sigma(a)}) \ | \\
        & \quad \quad \quad \quad \quad \quad \quad \quad \quad \quad \quad \quad \quad (\cm : \oto{a} ) \in \overline{\lambda} \}
    \end{align*}
\end{frame}

\begin{frame}[fragile]
    \frametitle{Example}
    \begin{minted}[escapeinside=||]{java}
        |\pink{class} \green{Pair}|<|\green{X} \pink{extends} \green{Object}|<>,
                   |\green{Y} \pink{extends} \green{Object}|<>> |\pink{extends} \green{Object}|<>{
          |\green{X} fst|;
          |\green{Y} snd|;
    
          setfst(newfst) {
            |\pink{return new} \green{Pair}|(|newfst|, |this.snd|);
          }
        }
    \end{minted}
\end{frame}

\begin{frame}[fragile]
    \frametitle{Example}
    \begin{align*}
        \lambda &= \{ ( \cxnm{Pair}{X}{Object}{setfst} ): \ [\text{a}_1] \to \text{a}_0 \}\\
        \text{C} &= \{ \text{a}_0 < \text{Object}, \ \text{a}_1 < \text{Object} \}\\
    \end{align*}

    \begin{align*}
        \text{C} = \{& \text{a}_0 < \text{Object}, \ \text{a}_1 < \text{Object}, \\
        &\text{a}_1 < \text{a}_5, \ \text{a}_5 < \text{Object}, \ \text{a}_2 < \text{a}_6, \ \text{a}_6 < \text{Object}, \\
        & \{ \{ \text{Pair<X, Y>} < \text{Pair<} \text{a}_3, \text{a}_4 \text{>}, \ \text{a}_2 = \text{a}_4, \ \text{a}_3 < \text{Object}, \ \text{a}_4 < \text{Object} \} \} \\
        &\}\\
    \end{align*}
\end{frame}

\begin{frame}[fragile]
    \frametitle{Example}
    \begin{align*}
        \lambda = \{& ( \cxnm{Pair}{X}{Object}{setfst} ): \ [\text{a}_1] \to \text{a}_0 \}\\
        \text{C} = \{& \text{a}_0 < \text{Object}, \ \text{a}_1 < \text{Object}, \\
        &\text{a}_1 < \text{a}_5, \ \text{a}_5 < \text{Object}, \ \text{a}_2 < \text{a}_6, \ \text{a}_6 < \text{Object}, \\
        & \{ \{ \text{Pair<X, Y>} < \text{Pair<} \text{a}_3, \text{a}_4 \text{>}, \ \text{a}_2 = \text{a}_4, \ \text{a}_3 < \text{Object}, \ \text{a}_4 < \text{Object} \} \},\\
        &\text{Pair<}\text{a}_5, \text{a}_6\text{>} < \text{a}_0\\
        &\}\\
    \end{align*}
\end{frame}

\begin{frame}[fragile]
    \frametitle{Example}
    \begin{align*}
        \frac{\cand \{ \nvtt{Pair}{\nvtt{X}{}, \nvtt{Y}{}} < \nvtt{Pair}{\fa{3}, \fa{4}} \} }{\cand \{ \nvtt{Pair}{\nvtt{X}{}, \nvtt{Y}{}} = \nvtt{Pair}{\fa{3}, \fa{4}} \} } \quad \text{adapt}
    \end{align*}

    \begin{align*}
        \frac{\cand \{ \nvtt{Pair}{\nvtt{X}{}, \nvtt{Y}{}} = \nvtt{Pair}{\fa{3}, \fa{4}} \} }{\cand \{ \nvtt{X}{} = \fa{3}, \ \nvtt{Y}{} = \fa{4} \} } \quad \text{reduce}
    \end{align*}

    \begin{align*}
        \frac{\cand \{ \nvtt{X}{} = \fa{3} \} }{\cand \{ \fa{3} = \nvtt{X}{} \} } \quad \text{and} \quad \frac{\cand \{ \nvtt{Y}{} = \fa{3} \} }{\cand \{ \fa{3} = \nvtt{Y}{} \} } \quad \text{swap}
    \end{align*}
\end{frame}

\begin{frame}[fragile]
    \frametitle{Example}
    \begin{align*}
        \text{C} = \{& \nvtt{Y}{} < \fa{6}, \ \fa{1} < \fa{5}, \ \fa{6} < \nvtt{Object}{}, \ \fa{1} < \nvtt{Object}{}, \\
        &\nvtt{X}{} < \nvtt{Object}{}, \ \fa{4} = \nvtt{Y}{}, \ \nvtt{Y}{} < \nvtt{Object}{}, \\
        &\fa{0} = \nvtt{Pair}{\fa{5}, \ \fa{6}}, \ \fa{3} = \nvtt{X}{}, \fa{2} = \nvtt{Y}{}, \ \fa{5} < \nvtt{Object}{} \\
        &\}
    \end{align*}

    \begin{align*}
        \text{C}_= &= \{ \fa{0} = \nvtt{Pair}{\fa{1}, \ \text{Y}}, \ \fa{6} = \text{Y}, \ \fa{5} = \fa{1}, \ \fa{4} = \text{Y}, \ \fa{3} = \text{X}, \ \fa{2} = \text{Y} \}\\
        \text{C}_< &= \{ \fa{1} < \nvtt{Object}{} \}
    \end{align*}
\end{frame}

\begin{frame}[fragile]
    \begin{align*}
        \text{Vielen Dank für Ihre Aufmerksamkeit!}
    \end{align*}
\end{frame}

\end{document}