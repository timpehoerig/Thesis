\begin{figure}[tp]
    \begin{align*}
        \text{T} &:= \text{X} \ | \ \text{N}\\
        \\
        \text{N} &:= \text{C} \vv{\ot{T}}\\
        \\
        \text{L} &:= \text{class} \ \text{C} \xn \ \vartriangleleft \ \text{D} \{ \ \ot{T} \ \ot{f} \ [\text{K}] \ \ot{M} \ \}\\
        \\
        \text{M} &:= \text{m}(\ot{x}) \ \{\text{return e}\}\\
        \\
        \text{e} &:= \text{x} \ | \ \text{e.f} \ | \ \text{e.m(}\ot{e}\text{)} \ | \ \text{new C(}\ot{e}\text{)} \ | \ \text{(C)e}
    \end{align*}
    \caption{Syntax}
    \label{syntax}
\end{figure}

Figure \ref{syntax} defines the syntax of Featherweight Generic Java: types (\verb|T| and \verb|N|), class definitions (\verb|L|), method definitions (\verb|M|) and expressions (\verb|e|).

\subsection{Types}
Types (\verb|T|, \verb|U|, \verb|V|) in Featherweight Generic Java are either type parameters (\verb|X|, \verb|Y|, \verb|Z|) or types other than type parameters (\verb|N|, \verb|P|, \verb|Q|).
In the following $\ot{T}$ is written short for $\text{T}_1, \text{T}_2, ... \text{T}_n$.

\subsection{Class Definitions}

A class definition \verb|L| always begins with the keyword \verb|class| followed by its name. Then, every type parameter and its upper bound, which are used as a field type in the class definition, are declared within \verb|<>|. The symbol $\vartriangleleft$ is used as a shortcut for the keyword \verb|extends|.
This is followed by the name of the superclass indicated by another \verb|extends| in front of it. Inside the body of a class definition the fields $\ot{f}$ of the class are defined with their types in front of them, followed by the optional constructor \verb|K| and all method definitions $\ot{M}$.

\subsection{Method Definitions}

As this is a type inference algorithm, every type annotations of methods can be dropped. Leading a method definition only to consist of its name \verb|m|, its arguments $\ot{x}$ and a single expression \verb|e| which is returned.

\subsection{Expressions}

There are five different forms of an expression \verb|e|: a simple variable \verb|X|, a field lookup written \verb|e.f|, a method invocation that takes one expression for each argument, an object creation indicated by the keyword \verb|new| followed by the name of the new object along its arguments or a cast.
A cast binds less than any other expression. Method calls and object creation do not need any instantiation of their generic types.

\subsection{Type Annotations}

Every type annotation except for field types and their upper bounds in class definitions can be dropped.

\subsection{Constructors}
In Featherweight Java as well as in Featherweight Generic Java a constructor is always required. The constructor has always the same form.

\begin{align*}
    \text{K} &:= \text{C}(\ot{f}, \ \ot{g}) \ \{ \text{super}(\ot{g}); \ \text{this}.\ot{f}=\ot{f}; \}
\end{align*}

The name of the constructor is always the name of the class itself. For every field of the class the constructor takes exactly one argument which must have the same name as the field it belongs to.
The body of the constructor always consists of two parts. First, there is a call to \verb|super|, which calls the constructor of the superclass. Here, all arguments $\ot{g}$ for fields that are not defined in the current class are passed on.
Second, every other argument $\ot{f}$ is initialized in the form \verb|this.f=f|.

As a constructor does not contain any information that cannot be found somewhere else in the class definition, writing the constructor for every class definition is redundant and is dropped in this implementation.
