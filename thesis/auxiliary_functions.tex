In the following sections all auxiliary functions that are used in the typeinference algorithm are specified.

\subsection{Auxiliary Typing}

\subsection{Subsitution}

\begin{figure}[H]
    \begin{align*}
1&: \txn{T}{X}{N}
\\
\\
2&: \tsxsn{T}{X}{N}
\\
\\
3&: \tsxsns{T}{X}{N}
    \end{align*}
    \caption{Substitution}
    \label{substitution}
\end{figure}

The first rule (1) shows the standart substitution rule. In a non variable type N every variable type X is replaced by the given type T.
\\
The second rule (2) is short for \inl{[T_1/X_1, ... , T_n/X_n]N}.
\\
The third rule (3) is short for \inl{[T_1/X_1, ... , T_n/X_n]N_1, ... , [T_1/X_1, ... , T_n/X_n]N_n}.
\\
These rules are implemented in different functions using pattern matching on the different type types.
\subsection{Subtyping}

\begin{figure}[H]
    \begin{align*}
1&: \ \Delta \vdash \text{T} <: \text{T}
\\
\\
2&: \ \frac{\Delta \vdash \text{S} <: \text{T} \quad \Delta \vdash \text{T} <: \text{U}}{\Delta \vdash \text{S} <: \text{U}}
\\
\\
3&: \ \Delta \vdash \text{X} <: \Delta(\text{X})
\\
\\
4&: \ \frac{
    \begin{matrix}
        \classheader \ \{...\} \\
        \Delta \vdash \ot{T} ok \quad \Delta \vdash \ot{T} <: \tsxsns{T}{X}{N}
    \end{matrix}}
    {\Delta C<\ot{T}> \ ok}
    \end{align*}
    \caption{Subtyping}
    \label{subtyping}
\end{figure}

\subsection{Auxiliary Functions}
