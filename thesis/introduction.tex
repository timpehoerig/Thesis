\subsection{Type Systems}
Consider the following definition of the function \inl{increment} which takes an argument \inl{n},
increments it by one and returns the result.
\begin{minted}[escapeinside=||]{java}
  |\black{increment}|(n) {
    |\pink{return}| n + 1;
  }
\end{minted}
Calling this function with an \inl{Integer}, for example 1, returns 2.
However, calling it with a \inl{String} results in a type error, because addition between a \inl{String}
and an \inl{Integer} is not defined.
Errors such as that are easy to make but cause the program to crash at runtime.
With type systems it is possible to detect such errors at compile time and thus increases the quality of programs dramatically.
In order to detect such errors at compile time additional information is given to the function declaration.
Consider the previous example but now with type annotations.
\begin{minted}[escapeinside=||]{java}
  |\green{Integer} \black{increment}|(|\green{Integer}| n) {
    |\pink{return}| n + 1;
  }
\end{minted}
Writing the type \inl{Integer} in front of the function name indicates the function's return type.
The arguments are annotated by writing their types in front of each argument.
Now the type system knows of which type the argument must be and thus calling \inl{increment} with a \inl{String} results in a type error at compile time rather than at runtime.

\subsection{Type Inference}
In order to correctly typecheck a whole program, everything that has a type needs an explicit type annotated,
even if the type annotation seems to be redundant. For example an instantiation of a variable.
An other down sight might be, that programmers do not always annotate the most general type. Consider the function \inl{plus} (\inl{+}) without any type annotations this function works for all types that support the addition operator like \inl{Int}, \inl{Double} or \inl{Float}.
Annotation this \inl{plus} function with one of those types would mean it is not possible to call it with another one.\\
So having the advantages of a type system without being forced to write the additional information yourself would be optimal. Type inference does just this.
However type inference has its boundaries. Inferring every type without any information is often not just hard but impossible. Different languages have different restrictions to make global type inference possible.

\subsection{Featherweight Generic Java}
Java is one of the most popular programming languages world wide. Extending Java with new features and providing proofs of soundness becomes more and more difficult as Java becomes more and more complex.
To be able to provide proofs about complex programming languages often a smaller version of the programming languages is defined that contains less features but behaves similar.
Extending those smaller versions with new features leads to languages still small enough so providing proofs not only becomes doable but handy.
Featherweight Java is exactly that. A smaller version of Java. In fact it is so small almost all features are dropped even assignments. Making Featherweight Java a functional subset of Java.
Featherweight Java only contains classes with fields, methods and inheritance and five different forms of expressions: Variables, field lookups, method invocations, object creation and castings.
As Java itself is not a functional programming languages it is easy to see that Featherweight Java is not equivalent to Java but because of its similar behavior it is still possible to draw back conclusions to the full Java.

A typical Featherweight Java program may look like this:

\begin{minted}[escapeinside=||]{java}
  |\pink{class} \green{Pair} \pink{extends} \green{Object}|{
    |\green{Object} fst|;
    |\green{Object} snd|;

    |\green{Pair} \black{setfst}|(|\green{Object}| newfst) {
      |\pink{return new} \green{Pair}|(|newfst, this.snd|);
    }
  }

  (|\pink{new} \green{Pair}|(
    |\pink{new} \green{Pair}|(|\pink{new} \green{Object}|(), |\pink{new} \green{Object}|()),
    |\pink{new} \green{Object}|()
  ).|fst|).|fst|
\end{minted}

A Featherweight Java program always consists of two parts. First, one ore more class definitions and second, an expression to be evaluated.

As well as Java, Featherweight Java includes a type system. With the type annotations given in the class definitions Featherweight Java is able to typecheck the given expression.
In the example above the most inner expression that has a fixed type is the object creation of \inl{new Pair(new Pair(new Object(), new Object()), new Object())}. Its type is \inl{Pair}.
Then the field \inl{fst} is accessed. Because the field \inl{fst} of class \inl{Pair} is annotated with \inl{Object} the resulting type is \inl{Object} even if the expression itself is \inl{new Pair(new Object(), new Object())}.
Now trying to access the field \inl{fst} the second time would be possible because the expression it is called on is an object creation of \inl{Pair} but because its type is \inl{Object} this leads to a compile time error.

Instead of giving fields a concrete type like \inl{Object}, giving them a type parameter solves this problem.
Extending Featherweight Java with Generics leads to Featherweight Generic Java.
Type parameters are just like normal parameters but they range over types. In Featherweight Generic Java every type parameter is given an upper bound.
In the following example the type parameters \inl{X} and \inl{Y} are introduced both with the upper bound \inl{Object}.
That means both type parameters can range over any types that are a subtype of \inl{Object}.

Type parameters can also be used in method declarations to set relations between different arguments and or the return type.
When a class is instantiated the type parameters are also instantiated with a concrete type.

Rewriting the example above results in:

\begin{minted}[escapeinside=||]{java}
  |\pink{class} \green{Pair}|<|\green{X} \pink{extends} \green{Object}|<>,
             |\green{Y} \pink{extends} \green{Object}|<>> |\pink{extends} \green{Object}|<>{
    |\green{X} fst|;
    |\green{Y} snd|;

    <|\green{Z} \pink{extends} \green{Object}|<>> |\green{Pair}|<|\green{Z}|, |\green{Y}|> |setfst|(|\green{Z} newfst|) {
      |\pink{return new} \green{Pair}|(newfst, |this.snd|);
    }
  }

  (|\pink{new} \green{Pair}|<|\green{Pair}|<|\green{Object}|<>, |\green{Object}|<>>, |\green{Object}|<>>(
    |\pink{new} \green{Pair}|<|\green{Object}|<>, |\green{Object}|<>>(|\pink{new} \green{Object}|<>(), |\pink{new} \green{Object}|<>()),
    |\pink{new} \green{Object}|<>()
  ).|fst|).|fst|
\end{minted}

Now trying to access the field \inl{fst} results in the same expression but with type \inl{Pair<Object<>, Object<>>} that is why the second field lookup is successful.

Generics are very powerful and nice to have but at the same time make type annotations much more complicated and bigger.
Adding type inference to Featherweight Generic Java would make it possible to rewrite the example above as follows:

\begin{minted}[escapeinside=||]{java}
  |\pink{class} \green{Pair}|<|\green{X} \pink{extends} \green{Object}|<>,
             |\green{Y} \pink{extends} \green{Object}|<>> |\pink{extends} \green{Object}|<>{
    |\green{X} fst|;
    |\green{Y} snd|;

    |setfst|(newfst) {
      |\pink{return new} \green{Pair}|(newfst, |this.snd|);
    }

    (|\pink{new} \green{Pair}|(
      |\pink{new} \green{Pair}|(|\pink{new} \green{Object}|(), |\pink{new} \green{Object}|()),
      |\pink{new} \green{Object}|()
    ).|fst|).|fst|
  }
\end{minted}

Here almost every type annotation is dropped except for class headers and field types.
\\
\\
The following sections first describe Featherweight Generic Java and Global Type Inference formal as it is describe in the paper "Global Type Inference for Featherweight Generic Java"\cite{FGJ} by Stadelmeier et al. except for a few changes and then shows how the Global Type Inference Algorithm is implemented.
