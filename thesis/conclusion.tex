I have implemented the type inference algorithm for Featherweight Generic Java as described in the paper\cite{FGJ} given by Stadelmeier et al.
The implementation includes a parser, a constraint generator and a constraint solver as well as an algorithm to combine them.
The function \verb|FJType| that generates the constraints, is implemented and behaves as it is described in the paper.
The function \verb|Unify| that solves the constraints may produce more than one solution and thus is implemented as a generator that
yields one solution after another. From the outside \verb|Unify| behaves as it is described in the paper. However, the implementation differs slightly. Especially the handling of type variables as upper bounds. In the original definition the different solutions are not ordered in any way
and even are nondeterministic. In this implementation an ordering is introduced which orders the solutions from most to least specific.

To combine both of those functions the paper defines the function\\\verb|TypeInference|. However, this function only combines those functions
for one class.

In order to make the algorithm useable for more than one class I changed the function \verb|TypeInference| to not only consider one class but a whole
program. However, this is not complete. It implements a depth first algorithm through the possible solutions. However, the first solution found may not be a correct one. This can be solved by adding a type checker. If the solution type checks, it is the correct one. If not, the algorithm backtracks.

Both papers "Global Type Inference for Featherweight Generic Java"\cite{FGJ} and "Featherweight Java: A Minimal Core
Calculus for Java and GJ"\cite{FJ} define their own typing rules for a Featherweight Generic Java Program which I both implemented\footnote{The rules of "Global Type Inference for Featherweight Generic Java" are defined in FGJ\_typing\_1\\The rules of "Featherweight Java: A Minimal Core Calculus for Java and GJ" are defined in FGJ\_typing\_2}.
Combining either one with the type inference algorithm imposes different problems which remain to be solved.

The code\footnote{\url{https://github.com/Proglang-Uni-Freiburg/FGJ-inference}} of this implementation can be found online. It comes with a command line tool to run the algorithm on any file that contains a
Featherweight Generic Program.

A good next step would be to combine the algorithm with one of the type checkers.
