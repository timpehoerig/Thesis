The type inference algorithm looks at one class after another. Therefore all classes must be ordered in a way that method calls only call methods from classes defined before the current class.
However, this does not constrain the algorithm as the program where this is not the case can be transformed to a program where it is.
This transformation is shown in the paper [...].
\\
In the following it is assumed that a program always fulfills this condition.

The algorithm mainly consist of two parts:
First, constraint generation and second, constraint solving.
As solving the constraints can lead to multiple solutions for a single class, simply one solution is assumed. If the algorithm later fails, it backtracks and assumes the next solution.

\begin{figure}[H]
    \begin{align*}
        \text{FJTypeInference}&(\prod, \ \classheader \{ ... \}) = \\
        \text{let} (\overline{\lambda}, \ \text{C}) &= \text{FJType}(\prod, \classheader \{ ... \}) \\
        (\sigma, \yp) &= \text{Unify}(\text{C}, \overline{\text{X}} <: \overline{\text{N}}) \\
        \text{in} \prod \cup \ \{( & \cm \ : \ \yp \ \oto{\sigma(a)}) \ | \ (\cm : \oto{a} ) \in \overline{\lambda} \}
    \end{align*}
    \caption{Type Inference}
    \label{type_inference}
\end{figure}
