\subsection{Abstract Syntax Tree}

Having a program in plain text form is rather useless for running the type inference algorithm. Instead, having the program in a specific form where class definitions can be accessed and traversed is desirable.
Such a form is called an Abstract Syntax Tree short AST. Every node in this AST is represented by a dataclass in Python \footnote{FGJ\_AST.py}.

The entry point for full program is the class \verb|Program|.
\begin{minted}[escapeinside=||]{python}
    [...]

    ClassTable = dict[str, ClassDef]


    @dataclass
    class Program:
        CT: ClassTable,
        expression: Expression
\end{minted}

The class \verb|Program| has two fields: \verb|CT| and \verb|expression|. \verb|CT|, short for class table, is a mapping from every class name to its definition. \verb|expression| is an expression to be evaluated.

The implementation of types, class definitions, method definitions and expressions are straight forward adaptations of their formal definitions in the paper.

Here, an example for the class definition \verb|ClassDef|:
\begin{minted}[escapeinside=||]{python}
    @dataclass
    class ClassDef:
        name: str,
        superclass: Type,
        typed_fields: FieldEnv,
        methods: list[MethodDef]
\end{minted}

There is no field for a constructor because, as mentioned earlier, constructors are dropped in this implementation.

The implementations for types and expressions differs slightly. Because there are different types and different expressions for both, first a plain class with no fields is defined: \verb|Type| and \verb|Expressions|. All the different kinds of types inherit from \verb|Type| and all kinds of expressions from \verb|expression|. Thereby simulating the structure of algebraic data types.
As a result it is possible to match against types and expressions.

\begin{minted}[escapeinside=||]{python}
    @dataclass
    class Type:
        pass

    @dataclass
    class TypeVar(Type):
        ...
\end{minted}

Every Featherweight Generic Java program can be represented by an AST.

\subsection{Parsing}

Parsing the program code to an AST is performed by using the python parser library \href{https://lark-parser.readthedocs.io/en/stable/}{Lark}.
Lark is a powerful library that can parse any context-free grammar.

There are two parts when using Lark:
firstly, the definition of the grammar.
Because every Featherweight Generic Java program can be represented as an AST, the grammar rules are trivial.
First, a rule for identifiers is created. This representation allows to easily change what is allowed to be an identifier at any time later.
Then, for every class defined for the AST a rule is defined recursively.
Second, for every rule in the grammar a function is defined. These functions get the parsed rule as an argument and return an instantiation of the respective class for the AST.
The most outer rule is \verb|Program| which then returns the whole program represented by the AST, starting with the class \verb|Program|.

\begin{minted}[escapeinside=||]{python}
    # grammar rules

    identifier: ...

    variable: identifier

    [...]

    # function for shaping

    def variable(tuple_of_elements_of_variable_rule):
        (name, ) = tuple_of_elements_of_variable_rule
        return Variable(name)
\end{minted}
