\subsection{Abstract Syntax Tree}

Having a program in plain text form is rather useless for running the type inference algorithm. Instead, having the program in a specific form where classdefinitions can be accessed and went through is desirable.
Such a form is called an Abstract Syntax Tree short AST. Every node of this AST is represented by a dataclass in Python \footnote{FGJ\_AST.py}.

The entry point for every full program is the class \inl{Program}.
\begin{minted}[escapeinside=||]{python}
    [...]

    ClassTable = dict[str, ClassDef]


    @dataclass
    class Program:
        CT: ClassTable,
        expression: Expression
\end{minted}

The class \inl{Program} has two fields: \inl{CT} for Class Table a mapping from every class name to its definition. Second, \inl{expression} an expression to be evaluated.

The implementation of types, classdefinitions, methoddefinitions and expressions are straight forward adaptations of their formal definitions in the paper "Global Type Inference for Featherweight Generic
Java".

For example the implementation for classdefinition \inl{ClassDef}:
\begin{minted}[escapeinside=||]{python}
    @dataclass
    class ClassDef:
        name: str,
        superclass: Type,
        typed_fields: FieldEnv,
        methods: list[MethodDef]
\end{minted}

There is no field for a constructor because as mentioned earlier constructors are dropped in this implemenatation.

The implemenatations for types and expressions differs a bit. Because there are different types and different expressions, for both first a plain class with no fields is defined \inl{Type} and \inl{Expressions} from which then all the different kinds of types or expressions inherit.
As result it is possible to match against types and expressions.

\begin{minted}[escapeinside=||]{python}
    @dataclass
    class Type:
        pass

    @dataclass
    class TypeVar(Type):
        ...
\end{minted}

Every Featherweight Generic Java program can be represented by an AST.

\subsection{Parsing}

Parsing the programcode to an AST is done by using the python parser library \href{https://lark-parser.readthedocs.io/en/stable/}{Lark}.
Lark is powerful library that can parse any context-free grammar.

There are two parst when using Lark:
First, the definition of the grammar.
Because every Featherweight Generic Java program can now be represented as an AST the grammar rules are trivial.
First a rule for identifier is created. This representation allows to easily changed what is allowed to be an identifier at any time later.
Then for every class defined for the AST a rule is define recursively.
Scond, for every rule in the grammar a function is defined wich gets the parsed rule as an argument and returns a instanciation of the respective class of the AST.
The most outer rule is \inl{program} wich then returns the whole program represented by the AST starting with the class \inl{Program}.

\begin{minted}[escapeinside=||]{python}
    # grammar rules

    identifier: ...

    variable: identifier

    [...]

    # function for shaping

    def variable(tuple_of_elemts_of_variable_rule):
        (name, ) = tuple_of_elemts_of_variable_rule
        return Variable(name)
\end{minted}
